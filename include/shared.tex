% Smol footnotes
\setbeamerfont{footnote}{size=\tiny}


% Use Fira Code for code
\setmonofont[
  Contextuals={Alternate},
  Scale=MatchLowercase
]{Fira Code}

% Hyperlink style
\hypersetup{
    linkcolor=blue,
    filecolor=magenta,
    urlcolor=cyan,
}
\urlstyle{same}

% Piece-wise uncovering of elements in TikZ pictures
\tikzset{
  invisible/.style={opacity=0},
  visible on/.style={alt={#1{}{invisible}}},
  alt/.code args={<#1>#2#3}{%
    \alt<#1>{\pgfkeysalso{#2}}{\pgfkeysalso{#3}} % \pgfkeysalso doesn't change the path
  },
}

% Default listing style
\lstdefinestyle{default}{%
  basicstyle=\ttfamily\footnotesize,
  showspaces=false,
  showstringspaces=false,
  showtabs=false,
  tabsize=1,
  commentstyle=\color{black!60}
}
\lstset{style=default}

\definecolor{lavender}{RGB}{162,85,162}

\newcommand{\SyNEHole}[3]{\ensuremath{\textcolor{lavender}{\bm{\llparenthesis}}#1\textcolor{lavender}{\bm{\rrparenthesis}^{#2}_{#3}}}}
\newcommand{\SyEHole}[2]{\ensuremath{\SyNEHole{}{#1}{#2}}}

\newcommand{\SyTInt}{\ensuremath{\textsf{Int}}}
\newcommand{\SyTBool}{\ensuremath{\textsf{Bool}}}

\newcommand{\SyDArrow}{\ensuremath{~\Rightarrow~}}
\newcommand{\SyNDArrow}{\ensuremath{~\nRightarrow~}}
\newcommand{\SyArrow}{\ensuremath{~\to~}}
\newcommand{\SyBar}{\ensuremath{\vert~}}
\newcommand{\SyColon}{\ensuremath{~\textbf{:}~}}
\newcommand{\SyInn}{\ensuremath{~\textbf{in}~}}
\newcommand{\SyIn}{\ensuremath{\textbf{in}~}}
\newcommand{\SyLet}{\ensuremath{\textbf{let}~}}
\newcommand{\SyWith}{\ensuremath{~\textbf{with}~}}
\newcommand{\SyWrap}{\ensuremath{\textbf{wrap}}}

\newcommand{\SyCastL}{\ensuremath{\langle}}
\newcommand{\SyCastR}{\ensuremath{\rangle}}

\newcommand{\SyPlus}{\ensuremath{+}}
\newcommand{\SyTimes}{\ensuremath{*}}
\newcommand{\SyTrue}{\ensuremath{\textrm{true}}}

\newcommand{\Indent}{\ensuremath{\hspace{10pt}}}

\newcommand{\isConsistent}[2]{\ensuremath{#1 \sim #2}}
\newcommand{\isNotConsistent}[2]{\ensuremath{#1 \nsim #2}}

\newcommand{\CtxVar}{\ensuremath{\Gamma}}
\newcommand{\HoleCtxVar}{\ensuremath{\Delta}}

\newcommand{\hasType}[2]{\ensuremath{#1 : #2}}
\newcommand{\hasTypeCtx}[4]{\ensuremath{#1, ~#2 \vdash \hasType{#3}{#4}}}

\newcommand{\extendCtx}[3]{\ensuremath{#1 ; ~\hasType{#2}{#3}}}

\newcommand{\rulen}[1]{~~\text{(#1)}}
\newcommand{\judgment}[3]{\inferrule{#1}{#2}\rulen{\textsc{#3}}}
\newcommand{\judgbox}[1]{\noindent \fbox{$#1$}}

\newcommand{\pie}[2]{%
  \begin{tikzpicture}
    \draw (0,0) circle (#1); \fill[rotate=90] (#1,0) arc (0:#2:#1) -- (0,0) -- cycle;
  \end{tikzpicture}%
}

\newcommand{\SyReturn}{\ensuremath{\textbf{return}~}}
\newcommand{\SyCaseComplete}{\ensuremath{\textbf{casecomplete}~}}
\newcommand{\SyEmbed}{\ensuremath{\textbf{emb}}}
\newcommand{\SyProj}{\ensuremath{\textbf{proj}}}

\newcommand{\SortNComplete}{\ensuremath{\textsf{Completeness}}}
\newcommand{\SortNCompleteVar}{\ensuremath{k}}
\newcommand{\SortComplete}{\ensuremath{\textsf{}}}
\newcommand{\SortCompleteVar}{\ensuremath{\overline{k}}}
\newcommand{\SortTypCon}{\ensuremath{\textsf{Type}}}
\newcommand{\SortTypConVar}{\ensuremath{\tau}}
\newcommand{\SortTyp}{\ensuremath{\textsf{}}}
\newcommand{\SortTypVar}{\ensuremath{\overline{\tau}}}
\newcommand{\SortComp}{\ensuremath{\textsf{Composite}}}
\newcommand{\SortCompVar}{\ensuremath{c}}
\newcommand{\SortValue}{\ensuremath{\textsf{Value}}}
\newcommand{\SortValueVar}{\ensuremath{v}}
\newcommand{\SortHoleId}{\ensuremath{u}}

\newcommand{\CNC}{\ensuremath{\pie{0.3ex}{360}}}
\newcommand{\CNI}{\ensuremath{\pie{0.3ex}{0}}}
\newcommand{\CII}{\ensuremath{\pie{0.3ex}{180}}}

\newcommand{\TCHole}{\ensuremath{\SyEHole{}{}}}
\newcommand{\TCInt}{\ensuremath{\SyTInt}}
\newcommand{\TCBool}{\ensuremath{\SyTBool}}

\newcommand{\TIntNC}{\ensuremath{\TMk{\TCInt}{\CNC}}}
\newcommand{\TBoolNC}{\ensuremath{\TMk{\TCBool}{\CNC}}}

\newcommand{\TMk}[2]{\ensuremath{#1[#2]}}

\newcommand{\ENumLit}{\ensuremath{\underline{n}}}
\newcommand{\EBoolLit}{\ensuremath{\underline{b}}}
\newcommand{\ELet}[2]{\ensuremath{\SyLet #1 = #2}}
\newcommand{\EIn}{\ensuremath{\SyInn}}
\newcommand{\EInn}[1]{\ensuremath{\SyIn #1}}
\newcommand{\EReturn}[1]{\ensuremath{\SyReturn #1}}
\newcommand{\ECaseCompleteWith}[1]{\ensuremath{\SyCaseComplete #1 \SyWith}}
\newcommand{\ECaseCompleteBranch}[2]{\ensuremath{\SyBar #1 \SyArrow #2}}

\newcommand{\EWrapIntoNI}[1]{\ensuremath{\SyWrap^{\CNI}~ #1}}
\newcommand{\EWrapIntoII}[1]{\ensuremath{\SyWrap^{\CII}~ #1}}
\newcommand{\EEmbedNC}[1]{\ensuremath{\SyEmbed_{\CNC}~ #1}}
\newcommand{\EEmbedNI}[1]{\ensuremath{\SyEmbed_{\CNI}~ #1}}
\newcommand{\EProj}[2]{\ensuremath{\SyProj[#2]~ #1}}

\newcommand{\EPlusNC}[2]{\ensuremath{#1 \SyPlus_{\CNC} #2}}
\newcommand{\EPlusNI}[2]{\ensuremath{#1 \SyPlus_{\CNI} #2}}
\newcommand{\ETimesNC}[2]{\ensuremath{#1 \SyTimes_{\CNC} #2}}
\newcommand{\ETimesNI}[2]{\ensuremath{#1 \SyTimes_{\CNI} #2}}

\newcommand{\ETrue}{\ensuremath{\SyTrue}}
\newcommand{\EEHole}[2]{\ensuremath{\SyEHole{#1}{#2}}}

\newcommand{\EVarNamed}[2]{\ensuremath{t_{#1}^{{\color{gray}#2}}}}
